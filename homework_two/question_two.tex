\subsection{What is Cross Validation}
Cross Validation is a type of statistical analysis on how well a model can predict future outcomes with the given dataset. Through various methods that can be used in cross validation the average performance of the methods will give insight to the models prediction performance.
\subsection{How does Cross Validation Work}
There are a couple different ways that Cross Validation can be done but the basis of all of these methods are: 
\begin{itemize}
	\item Splitting up the data into three different sets.
	\item Set on part to the \textbf{Past Set}, this will be the largest set.  
	\item Set another part to the \textbf{Test Set}, this is the second to largest. 
	\item Set the last smallest part to be the "\textbf{Future Set}". 
\end{itemize}
\subsection{When to use Cross Validation}
Cross Validation is best used when the task is very interpolative in nature and it because a lot harder to extrapolate into the future with Cross Validation. Examples of good ones would be image voice and text classification. An example of a bad use case of cross validation would be with ballistic missile calculation. It would be really hard to teach the machine the laws of physics and we would want to actually hard code those values into such a model.
